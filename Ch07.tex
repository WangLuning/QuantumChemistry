\documentclass{article}
\usepackage{graphicx} % Required for inserting images
\usepackage{CJK}
\usepackage{amsmath}
\usepackage{mathtools}
\title{Quantum Chemistry by Levine}
\author{LuMg}
\date{Sep 2023}

\begin{document}

\maketitle

\section{Chapter 7 Theorems of Quantum Mechanics}
\textbf{7.1}\\
(a) True\\
(b) True\\
(c) False\\
\newline

\textbf{7.2}\\
$c$ is real number\\
\newline

\textbf{7.3}\\
(a) $<m|n> = \int f_m^*f_nd\tau$\\
$<n|m>* = (\int f_n^*f_m d\tau)^* = \int f_m^*f_n d\tau$\\
(b) $<f|B|g> = \int f^*Bgd\tau$\\
$<cf|B|g> = \int c^*f^*Bgd\tau = c^*\int f^*Bgd\tau = c^*<f|B|g>$\\
\newline

\textbf{7.4}\\
unity operator\\
\newline

\textbf{7.5}\\
if $B$ is Hermitian\\
$<f|B|g> = \int f^*Bg  = <g|B|f>^* = (\int g^*Bf)* = \int (Bf)^*g = <Bf|g>$\\
\newline

\textbf{7.6}\\
Given $\^{A}$ is Hermitian:\\
(a) $<f|A|g> = <g|A|f>^*$\\
$<f|cA|g> = c<f|A|g>$\\
$<g|cA|f>^* = (\int g*cAf)^* = c^*\int (Af)^*g = c^*<g|A|f>$\\
so the equation only holds when c is real number\\
(b) given $<f|A|g>$ and $<f|B|g>$\\
$<f|A+B|g> = \int f^* (A+B)g = \int f^*Ag + \int f^*Bg$\\
\newline

\textbf{7.7}\\
(a)verify that $\frac{d^2}{dx^2}$ is Hermitian\\
$\int f* \frac{d^2}{dx^2}g = f^*\frac{dg}{dx}|^{\inf}_{-\inf} - \int \frac{df^*}{dx}\frac{dg}{dx}$\\
$ = - \int \frac{df^*}{dx}\frac{dg}{dx} = \int \frac{d^2f}{dx^2}g$\\
from \textbf{7.6}\\
$-\frac{\hbar^2}{2m}$ is real number\\
(b)$<T_x> = \int \Psi^*(-\frac{\hbar^2}{2m}\frac{d^2}{dx^2})\Psi$\\
$=-\frac{\hbar^2}{2m}(\Psi \frac{d\Psi}{dx}|^{\inf}_{-\inf} - \int \frac{d\Psi^*}{dx}\frac{d\Psi}{dx})$\\
$=\frac{\hbar^2}{2m}\int \frac{d\Psi^*}{dx}\frac{d\Psi}{dx}$\\
(c)$<T> = <T_x> + <T_y> + <T_z>$\\
(d)Given each component nonnegative, $<T> >=0$\\
\newline

\textbf{7.8}\\
(a) $\int f^*\frac{dg}{dx} = f^*g|^{\inf}_{-\inf} - \int \frac{df^*}{dx}g$\\
there is an extra negative sign\\
not Hermitian\\
(b) $\int f^*\frac{idg}{dx} = \int \frac{df^*}{dx}g$\\
so $i\frac{d}{dx}$ is Hermitian\\
(c) $\frac{d^2}{dx^2}$ is Hermitian\\
(d) $i\frac{d^2}{dx^2}$ is not Hermitian\\
\newline

\textbf{7.9}\\
representing a physical quantity means is Hermitian operator\\
(a) false\\
(b) false\\
(c) true\\
(d) true\\
\newline

\textbf{7.11}\\
Given $<A^2> = \int \Psi^* A^2\Psi = \int \Psi^* A (A\Psi)$\\
then $f = \Psi, g = A\Psi$ for $<f|A|g>$\\
for Hermitian $A$:\\
$= \int (A\Psi)*(A\Psi) = \int |A\Psi|^2$\\
\newline

\textbf{7.12}\\
(a) given $A$ and $B$ are Hermitian\\
$\int f^*ABg = \int (Af)*Bg = \int (BAf)*g$\\
if $AB$ is also Hermitian:\\
$\int f^*ABg = \int (ABf)*g$\\
So we need $AB = BA$, which means $A,B$ commute\\
(b)$\int f^*(AB+BA)g = \int f^*ABg + \int f^*BAg$\\
(c) both $x$ and $p_x$ are Hermitian\\
but they do not commute due to uncertainty\\
So $xp_x$ not Hermitian\\
(d) $\frac{1}{2}(xp_x+p_xx)$ is Hermitian\\
\newline

\textbf{7.13}\\
(a)$\int f^*\frac{dg}{dx} = f^*g|^{\inf}_{-\inf} - \int \frac{df^*}{dx}g = -<g|\frac{d}{dx}|f>$\\
(b)$\int f^*(AB-BA)g = \int f^*ABg - \int f^*BAg = \int (Af)^*Bg - \int (Bf)^*Ag = \int (BAf)^*g - \int (ABf)^*g = <g|BA|f>^* - <g|AB|f>^* = -<g|(AB-BA)|f>^*$\\
So $AB-BA$ is anti-Hermitian\\
\newline

\textbf{7.15}\\
(a) the eigenfunctions are orthogonal\\
given $\^{H}\Psi = E\Psi$ at stationary wave state:\\
$\^{H}\Psi = (n+\frac{1}{2})hv \Psi$\\
$<\Psi|\^{H}|\Psi> = (n + \frac{1}{2})hv\delta_{mn}$\\
(b) the eigenfunctions are orthogonal\\
given $\^{H}\Psi = E\Psi$ at stationary wave state:\\
$\^{H}\Psi = \frac{n^2h^2}{8ml^2}\Psi$\\
$<\Psi|\^{H}|\Psi> = \frac{n^2h^2}{8ml^2}\delta_{mn}$\\
\newline

\textbf{7.16}\\
$<\Psi_2|H|f(x)> = <f(x)|H|\Psi_2>^*$\\
where $H\Psi_2 = (2 + \frac{1}{2})hv\Psi$\\
the above equation becomes:\\
$= \frac{5}{2}hv<f(x)|\Psi_2>^*$\\
\newline

\textbf{7.19}\\
given the simulation equation:\\
$f = \Sigma<g_i|f>g_i$\\
given $g_i = \sqrt{\frac{2}{l}}sin(\frac{n\pi x}{l})$\\
so $<g_i|f> = \int_0^{l/2}g_i(-1)dx+\int_{l/2}^l g_idx$\\
\newline

\textbf{7.20}\\
(a) F. \\
like $Y^0_0$\\
(b) F.\\
only a complete set of eigenfunctions.\\
(c) T. \\
$L_z$ and $H$ commute.\\
\newline

\textbf{7.21}\\
if m is odd, $\prod^m = \prod$\\
if m is even, $\prod^m = \^{1}$\\
\newline

\textbf{7.22}\\
(a) s has the wave function like a sphere as even function\\
(b) $2p_x$ is odd function.\\
(c) $2s+2p_x$ are linear combination of $\^{H}$ under the same energy level\\
but $2s$ is even parity and $2p_x$ is odd parity, so they are not under the same eigenvalue of $\prod$\\
\newline

\textbf{7.23}\\
if $m != n$ $\psi_m$ and $\psi_n$ are orthogonal\\
$\int \psi^*_m \prod \psi_n = 0$\\
if $m = n$\\
since vibration wave function is either an odd or even function\\
$\prod$ has eigenvalue as $+1 or -1$\\
\newline

\textbf{7.24}\\
(a) $<2s|x|2p_x>$ since $\^{x}2p_x$ is even, so when it integrate with 2s, the result is an even function\\
integral not zero\\
(b) $<2s|x^2|2p_x>$ since both are odd function, integral is zero\\
(c) $<2p_y|x|2p_x>$ is an odd function, integral is zero\\
\newline

\textbf{7.25}\\
same as calculation of $\prod$\\
given $\^{R}f = rf$\\
$\^{R}^nf = r^nf = f$\\
then $r^n = 1$\\
\newline

\textbf{7.26}\\
prove that $\prod$ is both linear and Hermitian\\
(a)$\prod (f(x) + g(x)) = f(-x) + g(-x) = \prod f(x) + \prod g(x)$\\
(b)$<f(x)|\prod|g(x)> = \int f(x)^*\prod g(x) = \int f(-x)^*\prod g(-x)d(-x) = \int f(-x)^*g(x)d(-x) = \int (\prod f(x))^* g(x)dx$\\
\newline

\textbf{7.27}\\
intuitively, $\prod f = +1 f$ so $f$ is even function\\
$\prod g = -1 g$ so $g$ is odd function\\
then $\int f*g = 0$ is an odd function\\
\newline

\textbf{7.28}\\
$<v_1|x|v_2>$ where $v_1$ and $v_2$ are either odd or even according to quantum number\\
so if $v_1$ and $v_2$ both odd, or both even, integral is zero\\
otherwise, not zero\\
\newline

\textbf{7.29}\\
(a) do not need to calculate, can just imagine in 3-d cartisan coordinate.\\
$r = \sqrt{x^2+y^2+z^2}$ does not change\\
$\theta = \pi - \theta$\\
$\phi = \phi + \pi$\\
(b)$\prod e^{im\phi} = e^{im{\phi + \pi}} = e^{im\phi}e^{im\pi} = (-1)^{m}e^{im\phi}$\\
\newline

\textbf{7.30}\\
$\int\int\dots(\int f(q_1,q_2,\dots,q_m)dq_1dq_2\dots dq_k)\dots dq_m = 0$\\
\newline

\textbf{7.32}\\
the prob of getting $L_z = \hbar$ is $|\frac{1}{\sqrt{6}}|^2+|\frac{1}{\sqrt{3}}|^2 = \frac{1}{2}$\\
the prob of getting $L_z = 0$ is $|\frac{1}{\sqrt{3}}|^2 = \frac{1}{3}$\\
so the average $<L_z> = \frac{1}{2}\hbar$\\
\newline

\textbf{7.33}\\
The $p_0$ and $p_1$ state has $L^2 = 2\hbar^2$\\
The $d_0$ has $L^2 = 2*3\hbar^2$\\
$<L^2> = (\frac{1}{6}+\frac{1}{2})*2\hbar+\frac{1}{3}*6\hbar = \frac{10}{3}\hbar$\\
\newline

\textbf{7.34}\\
for both $p$ state:\\
$E_1 = -\frac{e^2}{4\pi \epsilon_0 8a}$\\
for $d$ state:\\
$E_2 = -\frac{e^2}{4\pi \epsilon_0 18a}$\\
So $<E> = (\frac{1}{6}+\frac{1}{2})E_1 + \frac{1}{3}E_2$\\
\newline

\textbf{7.35}\\
Given $L^2 = 2\hbar$\\
So $l= 1$\\
then $m = -1, 0, 1$\\
So the measurement of $L_x$ is $-\hbar, 0, \hbar$\\
\newline

\textbf{7.36}\\
the first term is $n = 1$ stationary state\\
the second term is $n = 2$ stationary state\\
So $\frac{1}{4}E_{n = 1} + \frac{3}{4}E_{n = 2} = \frac{13h^2}{32ml^2}$\\
\newline

\textbf{7.37}\\
Given the wave function $g_i = \sqrt{\frac{2}{l}}sin(\frac{n\pi x}{l})$\\
any non-stationary can be written as its combination\\
So each index, i.e. possibility is: \\
$|c_i^2| = \int g_i^*\Psi$\\
\newline

\textbf{7.38}\\
Shown as above:\\
$|c_i^2| = \int g_i^*\Psi$\\
\newline

\textbf{7.42}\\
(a)1\\
(b)0\\
(c)1\\
(d)0\\
\newline

\textbf{7.43}\\
$\int_{-\infty}^{\infty}\delta(x-a)dx = \int_{-\infty}^{\infty}\delta(x)^2dx$\\
$=\delta (0)\int_{-\infty}^{\infty}\delta(x)dx = \inf$\\
\newline

\textbf{7.44}\\
$\int_0^{\infty} f(x)\delta(x)$ is not the same as $\int_{-\infty}^{\infty} f(x)\delta(x)$\\
$\int_0^{\infty} f(x)\delta(x) = f(x)H(x)|_0^{\infty} - \int_0^{\infty} f^,(x)H(x)dx $\\
$= f(\infty) - \frac{1}{2}f(0) - f(x)|_0^{\infty} = \frac{1}{2}f(0)$\\
\newline

\textbf{7.49}\\
(a)\begin{pmatrix}
6&2\\
-12&-12\\
\end{pmatrix}
\\
(b)\begin{pmatrix}
    2&4\\
    8&-8\\
\end{pmatrix}\\
(c)\begin{pmatrix}
    3&0\\
    4&1\\
\end{pmatrix}\\
(d)\begin{pmatrix}
    6&3\\
    0&-9\\
\end{pmatrix}\\
(e)\begin{pmatrix}
    -2&5\\
    -16&-19\\
\end{pmatrix}\\
\newline

\textbf{7.50}\\
$CD = $\\
\begin{pmatrix}
    5i&10&5\\
    0&0&0\\
    -i&-2&-1\\
\end{pmatrix}\\
$DC=$\\
\begin{pmatrix}
    5i-1
\end{pmatrix}\\
\newline

\textbf{7.51}\\
all the unity orthogonal vector\\
\newline

\textbf{7.52}\\
$<f_i|P|f_j> = <f_i|sC|f_j> = c<f_i|C|f_j>$\\
i.e. $P_{ij} = sC_{ij}$\\
\newline

\textbf{7.53}\\
Given $<f_i|A|f_j> = a_i\delta_{ij}$\\
Given $f_i$ is a complete orthogonal set\\
Let $Af_j = \Sigma_k c_kf_k$\\
$<f_i|A|f_j> = \Sigma_k c_k <f_i|f_k> = \Sigma_k c_k \delta_{ik} = c_i$\\
So:\\
$Af_j = \Sigma_k <f_k|A|f_j>f_k = \Sigma_k a_i\delta_{ij}f_k = a_if_j$\\
So $a_i$ is the eigenvalue.\\
\newline

\textbf{7.63}\\
Given Hermitian:\\
$<f_i|A|f_j> = <Af_i|f_j>$\\
$\int f_i^*Af_j = \int (Af_i)*f_j$\\
So $A$ has eigenvalue $a$ has to be real\\
\newline

\textbf{7.64}\\
(a)F.\\
It can be a linear combination of time term prod stationary wave function\\
(b) T.\\
(c) F.\\
needs to be a stationary function\\
(d) F.\\
These eigenfunctions need to have the same eigenvalue\\
(e) F.\\
The measurement has to be a eigenvalue.\\
(f) T.\\
$|\Psi|^2$ is independent of time, which is what stationary means\\
(g) F.\\
They can have one common eigenfunction, but not a common complete set of eigenfunction.\\
(h) F.\\
They can have the same eigenfunction under different eigenvalue\\
(i) F.\\
It does not ensure they are non-degenerate\\
(j) F.\\
(k) T.\\
(l) F.\\
It does not have a fixed unit.\\
(m) F.\\
$<\Psi|A|\Psi>$ is real does not ensure $\Psi$ is real\\
(n) T.\\
(o) T\\
(p) F.\\
It holds for all well-behaved functions as long as $B$ is Hermitian\\
(q) T.\\
(r) F.\\
only for stationary states\\
\newline

\end{document}