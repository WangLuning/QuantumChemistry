\documentclass{article}
\usepackage{graphicx} % Required for inserting images
\usepackage{CJK}
\usepackage{amsmath}
\usepackage{mathtools}
\title{Quantum Chemistry by Levine}
\author{LuMg}
\date{Sep 2023}

\begin{document}

\maketitle

\section{Chapter 3 Operators}
\textbf{3.1}\\
(a)$\frac{d}{dx}cos(x^2+1) = -2xsin(x^2+1)$\\
(b)$5sinx$\\
(c)$sin^2x$\\
(d)$x$\\
(e)$\frac{-1}{x^2}$\\
(f)$24x+36x^3$\\
(g)$2xycos(xy^2)$\\
\newline

\textbf{3.2}\\
(a)operator\\
(b)function\\
(c)function\\
(d)operator\\
(e)operator\\
(f)function\\
\newline

\textbf{3.3}\\
$3x^2+\frac{d}{dx}$\\
\newline

\textbf{3.4}\\
(a) $1$\\
(b) $\frac{d}{dx}$\\
(c) $\frac{d^2}{dx^2}$\\
\newline

\textbf{3.5}\\
(a)$\frac{d}{dx}$\\
(b)$\frac{x}{2}$\\
(c)$\frac{1}{x^2}$\\
\newline

\textbf{3.6}\\
$(\^{A}+\^{B})f = \^{A}f + \^{B}f$\\
$(\^{B}+\^{A})f = \^{B}f + \^{A}f$\\
So:\\
$\^{A}+\^{B} = \^{B}+\^{A}$\\
\newline

\textbf{3.7}\\
$(\^{A}+\^{B})f = \^{A}f + \^{B}f = \^{C}f$\\
$\^{A}f = \^{C}f - \^{B}f$\\
$\^{A}f = (\^{C} - \^{B})f$\\
\newline

\textbf{3.8}\\
(a)$20x^3$\\
(b)$6x^3$\\
(c)$\frac{d^2}{dx^2}(x^2f(x))$\\
(d)$x^2\frac{d^2}{dx^2}f(x)$\\
\newline

\textbf{3.9}\\
$\^{A}\^{B} = x^3\frac{d}{dx}$\\
$\^{B}\^{A} = 3x^2 + x^3\frac{d}{dx}$\\
\newline

\textbf{3.11}\\
(a)$(\^{A}+\^{B})^2 = \^{A}^2 + \^{A}\^{B}+ \^{B}\^{A} + \^{B}^2 = (\^{B}+\^{A})^2$\\
(b) interchangable of $\^{A}$ and $\^{B}$\\
\newline

\textbf{3.12}\\
$[\^{A},\^{B}] = \^{A}\^{B}-\^{B}\^{A}$\\
$[\^{B},\^{A}] = \^{B}\^{A}-\^{A}\^{B}$\\
\newline

\textbf{3.13}\\
(a) $[sinz, \frac{d}{dz}] = sinz\frac{d}{dz} - cosz - sinz\frac{d}{dz}$\\
(b) $[\frac{d^2}{dx^2}, ax^2+bx+c]$\\
(c) $[\frac{d}{dx}, \frac{d^2}{dx^2}] = \frac{d^3}{dx^3} - \frac{d^3}{dx^3} = 0$\\
\newline

\textbf{3.14}\\
the def of linear operators: $T(ax+by) = Tax + Tby$\\
(a) linear.\\
(b) nonlinear.\\
$(a+b)^2 \neq a^2 + b^2$\\
(c) linear.\\
(d) nonlinear.\\
(e) linear.\\
(f) linear.\\
\newline

\textbf{3.16}\\
$(\^{A}\^{B})(af+bg) = \^{A}(\^{B}af+\^{B}bg) = \^{A}\^{B}af + \^{A}\^{B}bg$\\
\newline

\textbf{3.18}\\
(a)$\^{A}(bf + cg) = \^{A}bf + \^{A}cg = b\^{A}f + c\^{A}g$\\
(b)$\^{A}cg = c\^Ag$\\
\newline

\textbf{3.19}\\
(a) conjugation\\
$(a+b)^* = a^*+b^*$\\
but $(cg)^* = c^*g^* \neq cg^*$\\
\newline

\textbf{3.20}\\
(a) True, commutative\\
(b) False, has to be linear operator.\\
(c) False, left function, right operator.\\
(d) False, very strong $[\^{A},\^{B}] = 0$\\
(e) False, function and operator.\\
(f) False, linear operators satisfy.\\
(g) True, inside function commutative.\\
(h) True, both are functions.\\
\newline

\textbf{3.21}\\
(a)$\^{T}_c(af+bg) = af(x+c) + bg(x+c) = \^{T}_caf + \^{T}_cbg$\\
(b)$(x+2)^2 - 3(x+1)^2+2x^2$\\
\newline

\textbf{3.22}\\
$e^{\^{D}} = 1+\frac{d}{dx}+\frac{1}{2!}\frac{d^2}{dx^2}+\dots$\\
$\^{T}_1f(x) = f(x+1) = f(x) + 1*\frac{df(x)}{dx}+\dots$\\
which is the Taylor expansion.\\
\newline

\textbf{3.23}\\
(a) T.\\
(b) F.\\
(c) T.\\
(d) T.\\
(e) T.\\
\newline

\textbf{3.24}\\
(a)$(\frac{\partial ^2}{\partial x^2} + \frac{\partial ^2}{\partial y^2})(e^{2x}e^{3y}) = 4e^{2x}e^{3y}+9e^{2x}e^{3y} = 13e^{2x}e^{3y}$\\
(b)$(\frac{\partial ^2}{\partial x^2} + \frac{\partial ^2}{\partial y^2})(x^3y^3) = 6xy^3+6yx^3$\\
Not an eigenfunction.\\
(c)$(\frac{\partial ^2}{\partial x^2} + \frac{\partial ^2}{\partial y^2})(sin(2x)cos(4y)) = -4sin(2x)cos(4y)-16sin(2x)cos(4y) = -20sin(2x)cos(4y)$\\
eigenvalue -20.\\
(d)$(\frac{\partial ^2}{\partial x^2} + \frac{\partial ^2}{\partial y^2})(sin(2x)+cos(3y)) = -4cos(2x) - 9cos(3y)$\\
Not an eigenfunction.\\
\newline

\textbf{3.25}\\
find the value of the differential equation:\\
$-\frac{\hbar^2}{2m}s^2 = k$\\
$s = i\sqrt{2mk}/\hbar$\\
$\Psi = Ae^{sx}+Be^{-sx}$\\
\newline

\textbf{3.26}\\
$y = Ae^{kx}$\\
k needs to be imaginary.\\
\newline

\textbf{3.30}\\
(a)$[\^{x},\^{p}_x] = x\frac{\hbar}{i}\frac{\partial}{\partialx} - \frac{\hbar}{i}\frac{\partial}{\partial x}x*$\\
(b)$[\^{x},\^{p}_x^2] = (\frac{\hbar}{i})^2[x\frac{\partial ^2}{\partial x^2} - \frac{\partial ^2}{\partial x^2}x*]$\\
(c)$[\^{x}, \^{p}_y] = \frac{\hbar}{i}(x\frac{\partial}{\partial y} - \frac{\partial}{\partial y}x) = 0$\\
(d)$[\^{x},V(x,y,z)] = xV - Vx = 0$\\
(e)$[\^{x},\^{H}] = x((-\frac{\hbar^2}{2m}\triangledown^2)+V) - ((-\frac{\hbar^2}{2m}\triangledown^2)+V)x$\\
(f)$[\^{x}\^{y}\^{z},\^{p}_x^2]$\\
\newline

\textbf{3.31}\\
$-\frac{\hbar^2}{2m_1}\triangledown_1^2 -\frac{\hbar^2}{2m_2}\triangledown_2^2$\\
\newline

\textbf{3.32}\\
$\^{H} = \frac{\hbar^2}{2m}\triangledown^2+c(x^2+y^2+z^2)$\\
\newline

\textbf{3.33}\\
(a)$\int_0^2|\Psi(x,t)|^2dx$\\
(b)$\int\int\int_0^2|\Psi(x,y,z,t)|^2dxdydz$\\
(c)$\int\int\int\int\int\int_0^2|\Psi(x_1,y_1,z_1,x_2,y_2,z_2)|^2dx_1dy_1dz_1dx_2dy_2dz_2$\\
\newline

\textbf{3.34}\\
$\Psi$ does not have a concrete meaning\\
$\Psi^2$ is the prob density\\
(a)$\int_0^2|\Psi(x,t)|^2dx$\\
$dx$ has SI of $m$\\
So $\Psi(x,t)$ has SI $m^{-\frac{1}{2}}$\\
(b)$\int\int\int_0^2|\Psi(x,y,z,t)|^2dxdydz$\\
$dxdydz$ has SI $m^3$\\
So $\Psi(x,t)$ has SI $m^{-\frac{3}{2}}$\\
(c)$\int\int\int\int\int\int_0^2|\Psi(x_1,y_1,z_1,x_2,y_2,z_2)|^2dx_1dy_1dz_1dx_2dy_2dz_2$\\
$dx_1dy_1dz_1 \dots dx_ndy_ndz_n$ has SI of $m^{3n}$\\
So  $\Psi(x,t)$ has SI $m^{-\frac{3n}{2}}$\\
\newline

\textbf{3.35}\\
the lowest lying excited state is $1,1,2$\\
which is degenerate.\\
ground state is $1,1,1,$\\
$E = h\nu = \frac{h^2}{8m}(\frac{1}{a^2}+\frac{1}{b^2}+\frac{2^2}{c^2} - \frac{1}{a^2} - \frac{1}{b^2} - \frac{1}{c^2})$\\
$\nu = 7.58*10^{14}s^{-1}$\\
\newline

\textbf{3.36}\\
(a)$Prob = \int_{a_0}^{a_1}\int_{b_0}^{b_1}\int_{c_0}^{c_1}|\Psi|^2dxdydz$\\
where in ground state where $n_x = n_y = n_z = 1$:\\
$F(x) = \sqrt{\frac{2}{a}}sin(\frac{\pi x}{a})$\\
$G(y) = \sqrt{\frac{2}{b}}sin(\frac{\pi y}{b})$\\
$H(z) = \sqrt{\frac{2}{c}}sin(\frac{\pi z}{c})$\\
$\Psi = F(x)G(y)H(z)$\\
(b) the same method as above\\
(c) the same method as above\\
\newline

\textbf{3.37}\\
$\^{p}_x = \frac{h}{i}\frac{\partial}{\partial x}$\\
(a) F.\\
(b) T.\\
(c) T.\\
(d) F.\\
\newline

\textbf{3.39}\\
(a) $n_x = 1, n_y = 1, n_z = 1$\\
So according to nodal graph, max at $(\frac{a}{2}, \frac{b}{2}, \frac{c}{2})$\\
(b) $n_x = 2, n_y = 1, n_z = 1$\\
max at $(\frac{a}{4}, \frac{b}{2}, \frac{c}{2})$ or $(\frac{3a}{4}, \frac{b}{2}, \frac{c}{2})$\\
\newline

\textbf{3.40}\\
$\int\int\intF(x)G(y)H(z)dxdydz = \int\intGH(\intFdx)dydz = \intFdx\intGdy\intHdz$\\
\newline

\textbf{3.41}\\
$E = (\frac{n_x^2}{a^2}+\frac{n_y^2}{b^2}+\frac{n_z^2}{c^2})\frac{\h^2}{8m}$\\
So as long as:\\
$(\frac{n_x^2}{a^2}+\frac{n_y^2}{b^2}+\frac{n_z^2}{c^2}) = k$\\
degeneracy.\\
\newline

\textbf{3.42}\\
$(-\frac{\hbar^2}{2m}\triangledown^2+V)\Psi = E\Psi$\\
\newline

\textbf{3.43}\\
(a)$(\frac{n_x^2}{a^2}+\frac{n_y^2}{b^2}+\frac{n_z^2}{c^2}) = \frac{1^2+3^2+8^2}{l^2}$\\
So it is degenerate.\\
(b) not degenerate.\\
(c) it is degenerate.\\
\newline

\textbf{3.44}\\
the degenerate combinations are:\\
$n_xn_yn_z = (1,1,1), (1,1,2), (1,2,2), (2,2,2), (1,1,3), (1,2,3)$\\
6 different energy levels\\
17 different states (combinations).\\
\newline

\textbf{3.45}\\
(a) the only combination is $2,2,2$ not degenerate.\\
(b) the energy level comes from $1,2,3$ so 6 different states.\\
(c) the energy level comes from $1,1,5$ or $3,3,3$ so 4 different states.\\
\newline

\textbf{3.46}\\
(a)T.\\
(b)F.\\
(c)T.\\
(d)T.\\
(e)F.\\
(f)F.\\
(g)T.\\
\newline

\textbf{3.47}\\
$\^{H}(\psi_1+\psi_2) = \^{H}\psi_1+\^{H}\psi_2$\\
\newline

\textbf{3.49}\\
$<A+B> = \int\Psi^*(A+B)\Psi dx = \int\Psi^*A\Psi dx + \int\Psi^*B\Psi dx$\\
\newline

\textbf{3.50}\\
(a) $e^{-ax}$ goes to infinity when $x<0$\\
not well-behaved.\\
(b) well-behaved.\\
(c) well-behaved.\\
(d) well-behaved.\\
(e) not continuous at $x = 0$\\
\newline

\textbf{3.51}\\
$i\hbar\frac{\partial}{\partial t}\Psi = \^{H}\Psi$\\
it is a linear operator.\\
\newline

\textbf{3.53}\\
(a) T.\\
(b) F.\\
Get eigenvalues at different prob.\\
(c) F.\\
Not guarantee to have same eigenvalue.\\
(d) F.\\
(e) F.\\
not guarantee to have the same eigenvalue of E.\\
(f) F.\\
not guarantee to be stationary states.\\
(g) F.\\
(h) F.\\
(i) T.\\
adding time term it also holds.\\
(j) T.\\
(k) T.\\
(l) F.\\
(m) T.\\
$\^{A}\^{A}f = \^{A}(af) = a^2f$\\
(n) F.\\
order not exchangable.\\
(o) F.\\
\newline
\end{document}