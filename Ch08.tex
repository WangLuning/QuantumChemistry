\documentclass{article}
\usepackage{graphicx} % Required for inserting images
\usepackage{CJK}
\usepackage{amsmath}
\usepackage{mathtools}
\title{Quantum Chemistry by Levine}
\author{LuMg}
\date{Oct 2023}

\begin{document}

\maketitle

\section{Chapter 8 The Variation Method}
\textbf{8.1}\\
should be less than the ground state energy\\
$E \leq -203.2eV$\\
\newline

\textbf{8.2}\\
(a) Given only $V$ varies from $0$ to $l$:\\
$<\Psi|H|\Psi> = <\Psi|T|\Psi>+<\Psi|V|\Psi>$\\
where from PIB condition for ground state:\\
$<\Psi|T|\Psi> = \frac{h^2}{8ml^2}$\\
$<\Psi|V|\Psi> = \int_{l/4}^{3l/4}\Psi^*V_0\Psi dx = 0.818\frac{\hbar^2}{ml^2}$\\
So $<\Psi|H|\Psi> = 5.75\frac{\hbar^2}{ml^2}$\\
\newline
(b) similar as the method in (a)\\
$<\Psi|T|\Psi> = \int_0^l x(l-x)\frac{-\hbar^2}{2m}\frac{d^2}{dx^2}(x(l-x))dx = 0.16667\frac{\hbar^2l^3}{m}$\\
$<\Psi|V|\Psi> = \int_{l/4}^{3l/4}V_0x^2(l-x)^2dx = 0.0264\frac{\hbar^2l^3}{m}$\\
So $<\Psi|H|\Psi> = 0.193\frac{\hbar^2l^3}{m}$\\
And $\int_0^lx^2(l-x)^2dx = \frac{l^5}{30}$\\
then $W = 5.79\frac{\hbar^2}{ml^2}$\\
\newline

\textbf{8.3}\\
the idea is:\\
$<\Psi|H|\Psi> = <\Psi|T|\Psi>+<\Psi|V|\Psi>$\\
$= \int_0^{\infty}\Psi^*(\frac{-\hbar^2}{2m}\frac{d^2}{dx^2})\Psi dx +\int_0^{\infty}\Psi^*(2\pi^2 \nu^2m \Psi^2dx)$\\
\newline

\textbf{8.4}\\
this is not a well-behaved function\\
\newline

\textbf{8.5}\\
$\Psi = x(a-x)y(b-y)z(c-z)$\\
$\int \Psi^*\Psi d\tau = \frac{a^5}{30}\frac{b^5}{30}\frac{c^5}{30}$\\
$\int \Psi^*\^{H}\Psi d\tau$\\
According to variation method:\\
$\frac{\int \Psi^*\^{H}\Psi d\tau}{\int \Psi^*\Psi d\tau} = \frac{5h^2}{4\pi m}(\frac{1}{a^2}+\frac{1}{b^2}+\frac{1}{c^2})$\\
the true function is $\frac{h^2}{8m}(\frac{1}{a^2}+\frac{1}{b^2}+\frac{1}{c^2})$\\
the error is $1.3\%$\\
\newline

\textbf{8.6}\\
Given $\Psi = b-r$\\
$\int \Psi^*\Psi dtau = \int_0^{2\pi}\int_0^{\pi}\int_0^{\infty}(b-r)^2r^2sin\theta drd\theta d\phi = \frac{2\pi b^5}{15}$\\
$\int \Psi \^{H}\Psi d\tau = \int_0^{2\pi}\int_0^{\pi}\int_0^{b}(b-r)(-\frac{\hbar^2}{2m}(\frac{d^2}{dr^2}+\frac{2}{r}\frac{d}{dr})(b-r))dr d\theta d\phi = \frac{h^2b^3}{6\pi m}$\\
So:\\
$\frac{\int \Psi \^{H}\Psi d\tau}{\int \Psi^*\Psi dtau} = 0.1266\frac{h^2}{mb^2}$\\
Compared to the true function:\\
$E_1 = 0.125\frac{h^2}{mb^2}$\\
\newline

\textbf{8.7}\\
there is nothing to do with other info\\
$\frac{\partial W}{\partial c} = 0$\\
\newline

\textbf{8.8}\\
(a) $V = -\infty$ at $x<0$, so $V = 0$ when $x = 0$\\
(b) Given $\Psi = xe^{-cx}$\\
$<\Psi|\Psi> = \int_0^{\infty}x^2e^{-2cx}dx$\\
$<\Psi|V|\Psi> = \int_0^{\infty}bxx^2e^{-2cx}dx$\\
$<\Psi|T|\Psi> = \int_0^{\infty}xe^{-cx}(\frac{-\hbar^2}{2m}\frac{d^2}{dx^2})xe^{-cx}dx$\\
So: $W=\frac{3b}{2c}+\frac{\hbar^2c^2}{2m}$\\
find the minimum of this function at $\frac{\partial W}{\partial c} = 0$\\
$c = \frac{3bm}{2\hbar^2}^{1/3}$\\
So $W = 1.96\frac{b^2\hbar}{m}^{1/3}$\\
\newline

\textbf{8.9}\\
using normalization:\\
$\Psi_1 = N_1a(f+\frac{b}{a}g)$\\
$\Psi_2 = f+cg$\\
\newline

\textbf{8.10}\\
Given the variation function:$\Psi = e^{-cr}$\\
$\int \Psi^*\Psi d\tau = \int_0^{2\pi}\int_0^{\pi}\int_0^{\infty}e^{-2cr}r^2sin\theta drd\theta d\phi = \frac{\pi}{c^3}$\\
$\int \Psi^* \^{H}\Psi d\tau = \int e^{-cr}(\frac{-\hbar^2}{2\mu}-\frac{Ze^2}{4\pi \epsilon_0 r})e^{-cr}d\tau$\\
$W = \frac{\hbar^2c^2}{2\mu} - \frac{Ze^2c}{4\pi \epsilon_0}$\\
using $\frac{\partial W}{\partial c} = 0$\\
$c = \frac{Ze^2\mu}{4\pi \epsilon_0 \hbar^2}$\\
$W = \frac{Z^2e^4\mu}{2(4\pi \epsilon_0 \hbar)^2}$\\
there is no error\\
\newline

\textbf{8.11}\\
Guess $\Psi = e^{-bx^2}$\\
$\int \Psi^*\Psi d\tau = \sqrt{\frac{\pi}{2b}}$\\
where $H = T + V = \frac{-\hbar^2}{2m}\frac{d^2}{dx^2}+cx^4$\\
$\int \Psi^* \^{H}\Psi d\tau$\\
$W = \frac{\int \Psi^* \^{H}\Psi d\tau}{\int \Psi^*\Psi d\tau} = \frac{\hbar^2b}{2m}+\frac{3c}{16b^2}$\\
to find the minimum of W:\\
$\frac{\partial W}{\partial c} = 0$\\
$W = 0.681\hbar\frac{c\hbar}{m^2}^{1/3}$\\
\newline

\textbf{8.12}\\
Given $\Psi = x^k(l - x)^k$\\
$\int \Psi^*\Psi d\tau = i^{4k+1}\frac{\tau_{2k+1}^2}{\tau_{4k+2}}$\\
Since $V = 0$, $H = -\frac{\hbar^2}{2m}\frac{d^2}{dx^2}$\\
$\int \Psi^*\^{H}\Psi d\tau$\\
$W = \frac{\int \Psi^*\^{H}\Psi d\tau}{\int \Psi^*\Psi d\tau} = \frac{h^2(4k^2+k)}{4\pi^2 ml^2(2k-1)}$\\
To find optimal k:\\
$\frac{\partial W}{\partial k} = 0$\\
$k = 1.11$\\
$W = 0.1253\frac{h^2}{ml^2}$\\
\newline

\textbf{8.13}\\
Given the wave function $\Psi = sin(x\frac{x+c}{l+2c})$\\
We have $\int_{-c}^{l+c}\Psi^2dx = \frac{l+2c}{2}$\\
$<\Psi|V|\Psi> = \int_{-c}^0V_0\Psi^2dx + \int_l^{l+c}V_0\Psi^2dx$\\
$<\Psi|T|\Psi> = \int_{-c}^{l+c}\Psi(\frac{-\hbar^2}{2m}\frac{d^2}{dx^2})\Psi dx$\\
So $W = \frac{<\Psi|T|\Psi>+<\Psi|T|\Psi>}{<\Psi|\Psi>} = \frac{h^2}{8m(l+2c)^2}+\frac{2V_0c}{l+2c}-\frac{V_0}{\pi}sin(\frac{\pi l}{l + 2c})$\\
To minimize W using $\frac{\partial W}{\partial c} = 0$\\
\newline

\textbf{8.14}\\
the equation only holds when $E = E_1$\\
\newline

\textbf{8.16}\\
Given the triangular function:\\
$<\Psi|\Psi> = \int_0^{l/2}x^2dx+\int_{l/2}^l (l-x)^2dx$\\
$<\Psi|H|\Psi> = 0.152\frac{h^2}{ml^2}$\\
\newline

\textbf{8.17}\\
Given the wave function:\\
$\Psi = e^{-cr^2/a_0^2}$\\
Then:\\
$\int_0^{\infty}\Psi^*\Psi r^2dr = \frac{\pi}{2c}^{3/2}a_0^3$\\
where $H = T + V$ in spherical coordinate:\\
$T = -\frac{\hbar^2}{2m}(\frac{d^2}{dr^2}+\frac{2}{r}\frac{d}{dr})$\\
$V = -\frac{Ze^2}{4\pi \epsilon_0 r}$\\
So $\int \Psi^* \^{H}\Psi d\tau = \int \Psi (T+V)\Psi d\tau$\\
$W = \frac{\int \Psi^* \^{H}\Psi d\tau}{\int_0^{\infty}\Psi^*\Psi r^2dr} $\\
find the smallest $c$ that:\\
$\frac{\partial W}{\partial c} = 0$\\
$W = 0.4244\frac{Z^2e^2}{4\pi \epsilon_0 a_0}$\\
where the true wave function is $\Psi = 0.5\frac{Z^2e^2}{4\pi \epsilon a_0}$\\
the error is $15.1\%$\\
\newline

\textbf{8.19}\\
(a)\begin{vmatrix}
    3&1&i\\
    -2&4&0\\
    5&7&1/2\\
\end{vmatrix}
=i\begin{vmatrix}
    -2&4\\
    5&7\\
\end{vmatrix}+1/2\begin{vmatrix}
    3&1\\
    -2&4\\
\end{vmatrix}=7-34i\\
(b)\begin{vmatrix}
    2&5&1&3\\
    8&0&4&-1\\
    6&6&6&1\\
    5&-2&-2&2\\
\end{vmatrix}\\

\newline

\textbf{8.20}\\
(a) \begin{vmatrix}
    a_{11}&a_{12}&\dots&a_{1n}\\
    0&a_{22}&\dots&a_{2n}\\
    \dots\\
    0&\dots&\dots&a_{nn}\\
\end{vmatrix} = $a_{11}$ \begin{vmatrix}
    a_{22}&\dots&a_{2n}\\
    0&\dots&a_{33}&\dots\\
    \dots\\
    0&\dots&\dots&a_{nn}\\
\end{vmatrix} = $a_{11}a_{22}a_{33}\dots a_{nn}$\\
\newline

\textbf{8.21}\\
Given a 3-order determinant to try:\\
\begin{vmatrix}
    a_{11}&a_{12}&0\\
    a_{21}&a_{22}&0\\
    0&0&a_{33}\\
\end{vmatrix}
$=a_{11}a_{22}a_{33} - a_{12}a_{21}a_{33} = a_{33}(a_{11}a_{22} - a{12}a_{21})$\\
\newline

\textbf{8.23}\\
convert this into a determinant:\\
\begin{vmatrix}
    2&-1&4&2&16\\
    3&0&-1&4&-5\\
    2&1&1&-2&8\\
    -4&6&2&1&3\\
\end{vmatrix}\\
\newline

\textbf{8.24}\\
the accuracy of float number in computer\\
\newline

\textbf{8.26}\\
(a) \begin{vmatrix}
    8&-15\\
    -3&4\\
\end{vmatrix}$=77 >0$\\
So $x = 0, y = 0$\\
(b)\begin{vmatrix}
    -4&3i\\
    5i&\frac{15}{4}\\
\end{vmatrix}$ = 0$\\
$x = c, y = \frac{4}{3i}c$\\
\newline

\textbf{8.27}\\
(a)\begin{vmatrix}
    1&2&3\\
    3&1&2\\
    2&3&1\\
\end{vmatrix}=\begin{vmatrix}
    1&2&3\\
    0&-5&-7\\
    0&-1&-5\\
\end{vmatrix}=\begin{vmatrix}
    1&2&3\\
    0&-5&-7\\
    0&0&-\frac{18}{5}\\
\end{vmatrix} $!=0$\\
So $x = 0, y = 0, z = 0$\\
(b)\begin{vmatrix}
    1&2&3\\
    1&-1&1\\
    7&-1&11\\
\end{vmatrix} = \begin{vmatrix}
    1&2&3\\
    0&-3&-2\\
    0&-15&-10\\
\end{vmatrix} = \begin{vmatrix}
    1&2&3\\
    0&-3&-2\\
    0&0&0\\
\end{vmatrix}$ = 0$\\
So it has a non-trivial solution.\\
\newline

\textbf{8.29}\\
(a)F\\
(b)T\\
(c)F\\
(d)F\\
\newline

\textbf{8.30}\\
using the similar equation $det(H_{ij}-S_{ij}W) = 0$\\
\begin{vmatrix}
    4a-2bW&a-bW\\
    a-bW&6a-3bW\\
\end{vmatrix} = 0\\
Given two roots referring to $E_1$ and $E_2$:\\
$W_1 = 1.71a/b$ and $W_2 = 2.69a/b$\\
for example $E_1$ gives the evaluation:\\
$(4a-2bW_1)c_1+(a-bW)c_2 = 0$\\
$(a-bW_1)c_1+(6a-3bW)c_2 = 0$\\
and $<c_1f_1|c_2f_2> = 1$ as normalization:\\
$c_1 = 0.42\frac{1}{\sqrt{b}}, c_2 = 0.34\frac{1}{\sqrt{b}}$\\
similarly for $E_2$:\\
$(4a-2bW_2)c_1+(a-bW_2)c_2 = 0$\\
$(a-bW_2)c_1+(6a-3bW_2)c_2 = 0$\\
$c_1 = -0.85\frac{1}{\sqrt{b}}, c_2 = 0.70\frac{1}{\sqrt{b}}$\\
\newline

\textbf{8.31}\\
\begin{vmatrix}
    H_{11}-S_{11}W&H_{12} - S_{12}W\\
    H_{12} - S_{12}W&H_{11}-S_{11}W\\
\end{vmatrix} = 0\\
So $W = \frac{H_{11}+H_{12}}{S_{11}+S_{12}}$\\
Adding it back to the equation:\\
$\frac{c_1}{c_2} = 1 or -1$\\
\newline

\textbf{8.33}\\
Given $f_1 = x^2(l-x)$\\
$f_2 = x(l-x)^2$\\
So $S_11 = <f_1|f_1> = \frac{l^7}{105}$\\
$S_12 = <f_1|f_2> = \frac{l^7}{140}$\\
$S_22 = <f_2|f_2> = \frac{l^7}{105}$\\
$H_{11} = <f_1|H|f_1> = \frac{l^5\hbar^2}{15m}$\\
$H_{12} = <f_1|H|f_2> = \frac{l^5\hbar^2}{60m}$\\
$H_{22} = <f_2|H|f_2> = \frac{l^5\hbar^2}{15m}$\\
So $W_1 = \frac{5h^2}{ml^2}$\\
$W_2 = \frac{21h^2}{ml^2}$\\
\newline

\textbf{8.34}\\
Given the $x^, = x - \frac{l}{2}$\\
$f_1 = (\frac{l}{2}+x^,)(\frac{l}{2}-x^,)$\\
$f_2 = (\frac{l}{2}+x^,)^2(\frac{l}{2}-x^,)^2$\\
$f_3 = (\frac{l}{2}+x^,)(\frac{l}{2}-x^,)(-x^,)$\\
$f_4 = (\frac{l}{2}+x^,)^2(\frac{l}{2}-x^,)^2(-x^,)$\\
\newline

\textbf{8.35}\\
Given $f_1 = x(l-x)$\\
$f_2 = x^2(l-x)^2$\\
$H_{11} = <f_1|H|f_1> = \int_0^l x(l-x)\^{H}x(l-x)dx$\\
$H_{12} = <f_1|H|f_2> = \frac{\hbar^2l^5}{30m}$\\
$H_{22} = <f_2|H|f_2> = \frac{\hbar^2l^7}{105m}$\\
$S_{12} = <f_1|f_2> = \frac{l^7}{140}$\\
$S_{22} = <f_2|f_2> = \frac{l^9}{630}$\\
\newline

\textbf{8.36}\\
Given \begin{vmatrix}
    H_{33}-S_{33}W&H_{34} - S{34}W\\
    H_{43} - S_{43}W&H_{44} - S{44}W\\
\end{vmatrix} = 0\\
$W_1 = 0.5\frac{h^2}{ml^2}$\\
$W_2 = 2.54\frac{h^2}{ml^2}$\\
\newline

\textbf{8.41}\\
$A^* = $\begin{pmatrix}
    7&3&0\\
    2+i&-2i&-i\\
    1-i&4&2\\
\end{pmatrix}\\
$A^T = $\begin{pmatrix}
    7&2-i&1+i\\
    3&2i&4\\
    0&i&2\\
\end{pmatrix}\\
$A^\^ = $\begin{pmatrix}
    7&2+i&1-i\\
    3&-2i&4\\
    0&-i&2\\
\end{pmatrix}\\
\newline

\textbf{8.42}\\
(a)F\\
(b)CF\\
(c)DF\\
\newline

\textbf{8.45}\\
(a)\begin{vmatrix}
    0-\lambda&-1\\
    3&2-\lambda\\
\end{vmatrix} = 0\\
$\lambda_1 = 1+\sqrt{2}i, \lambda_2 = 1-\sqrt{2}i$\\
\begin{equation}
    -\lambda_1 c_1-c_2 = 0\\
    3c_1+(2-\lambda)c_2 = 0\\
\end{equation}
Resulting in the eigenfunction:\\
$c_1 = $\begin{pmatrix}
    \frac{1}{2}\\
    -\frac{1}{2}-\frac{1}{2}\sqrt{2}i\\
\end{pmatrix}\\
$c_2 = $\begin{pmatrix}
    \frac{1}{2}\\
    -\frac{1}{2}+\frac{1}{2}\sqrt{2}i\\
\end{pmatrix}\\
\newline
(b)\begin{vmatrix}
    2-\lambda&0\\
    9&2-\lambda
\end{vmatrix} = 0\\
So $\lambda = 2$\\
There is no determined eigenfunction.\\
\newline
(c)\begin{vmatrix}
    4-\lambda&0\\
    0&4-\lambda\\
\end{vmatrix} = 0\\
$\lambda = 4$\\
So there is no determined eigenfunction.\\
\newline

\textbf{8.46}\\
$(a_{11} - \lambda)(a_{22} - \lambda)(a_{33} - \lambda) = 0$\\
Generating three $\lambda = a_{11},a_{22},a_{33}$\\
Generating eigenfunction:\\
$c_1=$\begin{pmatrix}
    1\\
    0\\
    0\\
\end{pmatrix}\\
$c_2 =$\begin{pmatrix}
    0\\
    1\\
    0\\
\end{pmatrix}\\
$c_3 = $\begin{pmatrix}
    0\\
    0\\
    1\\
\end{pmatrix}\\
\newline

\textbf{8.47}\\
\begin{vmatrix}
    2-\lambda&2\\
    2&-1-\lambda\\
\end{vmatrix} = 0\\
$\lambda_1 = 3, \lambda_2 = -2$\\
So the first eigenfunction:\\
\begin{pmatrix}
    \frac{2}{\sqrt{5}}\\
    \frac{1}{\sqrt{5}}\\
\end{pmatrix}\\
the other eigenfunction is:\\
\begin{pmatrix}
    \frac{2}{\sqrt{5}}\\
    -\frac{1}{\sqrt{5}}\\
\end{pmatrix}\\
\newline

\textbf{8.66}
(a)True\\
(b)True\\
(c)True\\
(d)True\\
(e)True\\
(f)True\\
(g)True\\
(h) False\\
(i)True\\
(j)False\\
(k)False\\
(l)True\\
(m)True\\
(n)True\\
\newline

\end{document}