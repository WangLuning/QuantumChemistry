\documentclass{article}
\usepackage{graphicx} % Required for inserting images
\usepackage{CJK}
\usepackage{amsmath}
\usepackage{mathtools}
\title{Quantum Chemistry by Levine}
\author{LuMg}
\date{Sep 2023}

\begin{document}

\maketitle

\section{Chapter 2 The Particle in a Box}
\textbf{2.1}\\
(a)$s^2+s-6 = 0$\\
$y = c_1e^{-3x}+c_2e^{2x}$\\
(b) Given boundary condition:\\
$c_1+c_2=0$\\
$-3c_1+2c_2 = 1$\\
$c_1 = 0.2, c2=-0.2$\\
$y = 0.2e^{-3x}-0.2e^{2x}$\\
\newline

\textbf{2.2}\\
$(s-(2+i))*(s - (2-i)) = 0$\\
$s^2 - 4s + 5 = 0$\\
So the original equation is:\\
$\frac{d^2y}{dx^2}-4\frac{dy}{dx}+5 = 0$\\
\newline

\textbf{2.3}\\
(a) $(s-a)^2=0$\\
The original equation is:\\
$y^{,,} - 2ay^{,} + a^2y = 0$\\
The other root $xe^{ax}$:\\
$y^{,} = e^{ax}+axe^{ax}$\\
$y^{,,} = ae^{ax}+ae^{ax}+a^2xe^{ax}$\\
The original equation is:\\
$(2a+a^2x)e^{ax} - 2a*(1+ax)e^{ax}+ a^2xe^{ax} = 0$\\
holds\\
(b)$y^{,,} - 2y^{,}+y = 0$\\
$s = 1$\\
So according to (a) the equation is:\\
$y = c_1e^x+c_2xe^x$\\
\newline

\textbf{2.4}\\
$x = x(t)$\\
(a)$m\frac{d^2x}{dt^2} = c$\\
Linear.\\
(b)$m\frac{d^2x}{dt^2} = -kx$\\
Linear.\\
(c)$m\frac{d^2x}{dt^2} = ax^3$\\
Nonlinear.\\
(d)$m\frac{d^2x}{dt^2} = bsinax$\\
Nonlinear.\\
(e)$m\frac{d^2x}{dt^2} = a-kx$\\
Linear.\\
\newline

\textbf{2.5}\\
(a) False.
The energy level starts with $n=1$\\
(b) False.\\
Between two energy level is not evenly separated.\\
$E = \frac{n^2\hbar^2}{8ml^2}$\\
(c) True.\\
(d) False.\\
Not all time sensitive term is a phase term.\\
(e) True.\\
$n=1$ to $n=2$ needs least energy.\\
\newline

\textbf{2.6}\\
the state function is:\\
$\Psi = \sqrt{\frac{2}{l}}*sin(\frac{n\pi x}{l})$\\
So the wave form at different n is:\\
(a) max at $\frac{l}{2}$, min at $0$ and $l$\\
(b) max at $\frac{l}{4}$ or $\frac{3l}{4}$, min at $0$ or $\frac{l}{2}$ or $l$\\
(c) max at $\frac{l}{6}$ or $\frac{l}{2}$ or $\frac{5l}{6}$, min at $0$ or $\frac{l}{3}$ or $\frac{2l}{3}$ or $l$\\
\newline

\textbf{2.7}\\
(a) this is for all n value:\\
$\Psi = \sqrt{\frac{2}{l}}*sin(\frac{n\pi x}{l})$\\
$Prob = \int_0^{\frac{l}{4}}|\Psi|^2dx = \frac{1}{4} - \frac{n\pi}{2}sin(\frac{n\pi}{2})$\\
(b) maximize the above prob, when $n=3$\\
(c) $\frac{1}{4}$\\
(d) falls back to classic law\\
\newline

\textbf{2.8}\\
(a) this infinity is small enough:\\
$\Psi = \sqrt{\frac{2}{l}}*sin(\frac{n\pi x}{l})$\\
$Prob = |\Psi|^2dx = \frac{2}{l}sin^2(\frac{\pi x}{l})dx$\\
where:\\
$l = 2\r{A}$ and $x = 0.6\r{A}$ and $dx = 0.001\r{A}$\\
So:\\
$Prob = 6.65*10^{-4}$\\
in one million exps:\\
$n = 655$ times\\
(b) $Prob = |\Psi|^2dx = \frac{2}{l}sin^2(\frac{\pi x}{l})dx$\\
when $x = 0.7\r{A}$ and $x = 1.0\r{A}$\\
$\frac{Prob(0.7)}{Prob(1.0)} = \frac{126}{n}$\\
$n = 159$
\newline

\textbf{2.9}\\
(a)corresponding: 4 nodes and 5 nodes\\
(b) $\Psi = \sqrt{\frac{2}{l}}*sin(\frac{4\pi x}{l})$\\
$\Psi^2 = \frac{2}{l}sin^2(\frac{4\pi x}{l})$\\
$\frac{d\Psi^2}{dx} = \frac{4}{l}\frac{4\pi}{l}sin(\frac{4\pi x}{l})cos(\frac{4\pi x}{l})$\\
\newline

\textbf{2.10}\\
(a) Given PIB:\\
$E = \frac{n^2h^2}{8ml^2}$\\
So the diff for $n=1$ and $n=2$ is:\\
$E_{diff} = \frac{(2^2-1^2)*h^2}{8ml^2} = \frac{3*(6.626*10^{-34})^2}{8*9.109*10^{-31}*(10^{-10})^2} = 1.81*10^{-17}J$\\
(b) $E_{diff} = h\nu = h\frac{c}{\lambda}$\\
$\lambda = 110\r{A}$\\
(c) ultraviolet\\
\newline

\textbf{2.11}\\
it is a macro object, so we expect $n$ is very large.\\
When $n$ is large, the prob seems evenly distributed in the box.\\
$E = \frac{n^2h^2}{8ml^2}$\\
And in macro world:\\
$E = \frac{1}{2}mV^2$\\
So: $n = 3*10^{26}$\\
\newline

\textbf{2.12}\\
$E = h\nu = \frac{(5^2 - 2^2)h^2}{8ml^2}$\\
$l = 1.78nm$\\
\newline

\textbf{2.13}\\
$E = h\nu = h\frac{c}{\lambda} = \frac{(n^2 - 1)h^2}{8ml^2}$\\
$n = 4$\\
\newline

\textbf{2.14}\\
$\frac{E_1}{E_2} = \frac{\nu_1}{\nu_2} = \frac{2^2 - 1}{3^2 - 2^2}$\\
$\nu_2 = \frac{5}{3}&\nu_1 = 1*10^{13}s^{-1}$\\
\newline

\textbf{2.15}\\
$E = h\nu = \frac{n_a^2 - n_b^2}{8ml^2}h^2$\\
$n_a^2 - n_b^2 = 5$
So it is from $n = 2$ to $n = 3$\\
\newline

\textbf{2.16}\\
We need to find two sets of $n_a$ and $n_b$, where:\\
$n_a^2 - n_b^2 = k$ and $n_c^2 - n_d^2 = k$\\
an example is $k = 15$ for $(1,4)$ and $(7, 8)$\\
\newline

\textbf{2.17}\\
There are 4 pi electrons, taking $n=1$ and $n=2$ level.\\
So the excitement is from $n = 2$ to $n = 3$\\
$E = h\nu = \frac{(3^2-2^2)h^2}{8ml^2}$\\
$\nu = 320nm$\\
\newline

\textbf{2.18}\\
the boundary condition changes\\
inside the box:\\
$-\frac{\hbar^2}{2m}\frac{\partial^2}{\partial x^2}\Psi^2 = E\Psi$\\
$\Psi = Acos(\frac{\sqrt{2mE}}{\hbar}x) + Bsin(\frac{\sqrt{2mE}}{\hbar}x)$\\
boundar is:\\
$\Psi(x = -\frac{l}{2}) = 0$\\
and:\\
$\Psi(x = \frac{l}{2}) = 0$\\
if $A \neq 0$ then $B = 0$ and $\frac{\sqrt{2mE}}{\hbar}\frac{l}{2} = (k+\frac{1}{2})\pi$\\
then $E = \frac{(2k+1)^2h^2}{8ml^2}$\\
similarly:\\
if $B \neq 0$ then $A = 0$ and $\frac{\sqrt{2mE}}{\hbar}\frac{l}{2} = k\pi$\\
then $E = \frac{(2k)^2h^2}{8ml^2}$\\
combine them together:\\
$E = \frac{n^2h^2}{8ml^2}$\\
\newline

\textbf{2.20}\\
adding time term:
$E = e^{-iEt/\hbar}(Acos(\frac{\sqrt{2mE}}{\hbar}x) + Bsin(\frac{\sqrt{2mE}}{\hbar})x)$\\
\newline

\textbf{2.21}\\
(a) this is purely from boundary condition:\\
$\Psi_1(0) = \Psi_2(0)$\\
and $\Psi^,_1(0) = \Psi^,_2(0)$\\
(b) another boundary condition at $x = l$\\
\newline

\textbf{2.22}\\
$tan(\frac{\sqrt{2mE}}{\hbar}l) = -2\sqrt{\frac{E}{V_0}} = 0$\\
So:\\
$\frac{\sqrt{2mE}}{\hbar}l = n\pi$\\
it is the same as:\\
$E = \frac{n^2h^2}{8ml^2}$\\
\newline

\textbf{2.23}\\
$b = \sqrt{2mV_0}\frac{l}{\hbar} = 1.26\pi$\\
then $N - 1 < \frac{b}{\pi} <= N$\\
$N = 2$\\
\newline

\textbf{2.27}\\
$\epsilon = \frac{E}{V_0} = 0.15$\\
then $2(\epsilon - 1)sin(b\sqrt{\epsilon})-2\sqrt{\epsilon - \epsilon^2}cos(b\sqrt{\epsilon}) = 0$\\
$b = 6.07$\\
also $b = \sqrt{2mV_0}\frac{l}{\hbar}$\\
So: $l = 0.265nm$\\
\newline

\textbf{2.28}\\
$N = 1,2,3$\\
from $N - 1 < \frac{b}{\pi} <= N$\\
$2 < \sqrt{2mV_0}\frac{l}{\hbar\pi} <= 3$\\
So:\\
$l <= 5.20 \r{A}$\\
\newline

\textbf{2.29}\\
$N-1 < \sqrt{2mV_0}\frac{l}{\hbar\pi} <= N$\\
(a) increase with $V_0$\\
(b) increase with $l$\\
\newline

\textbf{2.32}\\
(a)F.\\
$n = 1,2,3..$ to make $\Psi \neq 0$\\
(b)F.\\
the wave function is continuous \\
(c) T.\\
for free one dimensional particle, since $\Psi_{\inf} is \inf$ so the derivitive is not continuous\\
(d)F.\\
Harmonical oscillation with odd n is 0 at center.\\
(e) T.\\
the prob densitiy func is an odd function.\\
(f) F.\\
the prob densitiy func is an even function.\\
(g) T.\\
needs more and more energy.\\
(h) F.\\
for macro object the prob density is constant, but not for micro particles.\\
(i) T.\\
classical object can have 0 energy. But micro object has $n>0$.\\
\newline
\end{document}