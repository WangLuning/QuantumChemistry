\documentclass{article}
\usepackage{graphicx} % Required for inserting images
\usepackage{CJK}
\usepackage{amsmath}
\usepackage{mathtools}
\title{Quantum Chemistry by Levine}
\author{LuMg}
\date{Sep 2023}

\begin{document}

\maketitle

\section{Chapter 5 Angular Momentum}
\textbf{5.1}\\
(a) no.\\
(b) yes\\
(c) yes\\
(d) yes\\
(e) yes\\
for all well-behaved functions\\
\newline

\textbf{5.2}\\
(a)$[\hat{A},\hat{B}] = \hat{A}\hat{B} - \hat{B}\hat{A} = -[\hat{B},\hat{A}]$\\
(b)$[\hat{A},\hat{A}^n] = \hat{A}^{n+1} - \hat{A}^{n+1} = 0$\\
(c)$[k\hat{A},\hat{B}] = k\hat{A}\hat{B} - \hat{B}k\hat{A} = k[\hat{A},\hat{B}]$\\
(d)$[\hat{A},\hat{B}+\hat{C}] = \hat{A}(\hat{B}+\hat{C}) - (\hat{B}+\hat{C})\hat{A} = [\hat{A},\hat{B}]+ [\hat{A},\hat{C}]$\\
(e)$[\hat{A},\hat{B}{C}] = \hat{A}\hat{B}\hat{C} - \hat{B}\hat{C}\hat{A}$\\
\newline

\textbf{5.3}\\
$[\hat{x},\hat{p_x}^3] = p_x[\hat{x},\hat{p_x}^2]+[\hat{x},\hat{p_x}]\hat{p_x}^2$\\
$=-i\hbar\frac{\partial}{\partial x}(2\hbar^2\frac{\partial}{\partial x})+i\hbar(-\hbar^2\frac{\partial^2}{\partial x^2})$\\
\newline

\textbf{5.4}\\
for a harmonic oscillator:\\
$<x> = 0$\\
for energy level $n = 0$, it has energy $E = \frac{hv}{2}$\\
from which $<T> = \frac{hv}{4}$\\
$<V> = \frac{1}{2}kx^2 = \frac{hv}{4}$\\
So $<x^2> = \frac{hv}{2k} = \frac{h}{8\pi^2\nu m}$\\
$(\triangle x)^2 = <x^2> - <x>^2 = \frac{h}{8\pi^2\nu m}$\\
So:\\
$\triangle x = \sqrt{\frac{h}{8\pi^2\nu m}}$\\
similarly:\\
$<p_x> = 0$\\
Given $<T> = \frac{p^2_x}{2m} = \frac{hv}{4}$\\
$<p^2_x> = \frac{hvm}{2}$\\
$\triangle p_x = \sqrt{<p^2_x> - <p_x>^2} = \sqrt{\frac{hvm}{2}}$\\
So:\\
$\triangle x \triangle p_x = \frac{h}{4\pi} = \frac{\hbar}{2}$\\
\newline

\textbf{5.5}\\
Given the non-stationary state:
$\Psi = \sqrt{105/l^7}x^2(l-x)$\\
We need to calculate 4 values: $<x>, <x^2>, <p_x>, <p^2_x>$\\
$<x> = \int_{-\inf}{\inf}\Psi^*x\Psi d\tau = \frac{5l}{8}$\\
$<x^2> = \int_{-\inf}^{\inf}\Psi^*x^2\Psi d\tau= \frac{5l^2}{12}$\\
$<p_x> = \int_{-\inf}^{\inf}\Psi^*(-i\hbar\frac{\partial}{\partial x})\Psi d\tau = 0$\\
$<p^2_x> = \int_{-\inf}^{\inf}\Psi^*(-i\hbar\frac{\partial}{\partial x})^2\Psi d\tau = \frac{14\hbar^2}{l^2}$\\
$\triangle x = \sqrt{<x^2> - <x>^2} = \sqrt{\frac{5}{192}}l$\\
$\triangle p_x = \sqrt{<p^2_x> - <p_x>^2} = \sqrt{14}\frac{\hbar}{l}$\\
So:\\
$\triangle x \triangle p_x = \sqrt{\frac{5}{192}}l \sqrt{14}\frac{\hbar}{l} > \frac{\hbar}{2}$\\
\newline

\textbf{5.6}\\
Given $\Psi$ is the eigenfunction, the only measurement value can be eigenvalue $a$\\
So it is expected there is no deviation.\\
$\triangle A = <A^2> - <A>^2$\\
where:\\
$<A^2> = \int_{-\inf}^{\inf}\Psi^*A^2\Psi = a^2$\\
$<A>^2 = a^2$\\
So:
$\triangle A = \sqrt{<A^2> - <A>^2} = 0$\\
\newline

\textbf{5.7}\\
$(\triangle A)^2 = \int_{-\inf}^{\inf}\Psi(A - <A>)^2\Psi d\tau$\\
$=\int_{-\inf}^{\inf}\Psi A^2 \Psi d\tau - 2<A>\int_{-\inf}^{\inf}\Psi A \Psi d\tau + <A>^2$
$=<A^2> - <A>^2$\\
\newline

\textbf{5.8}\\
All the combinations $HH, HT, TH, TT$\\
So $<w> = 1$\\
where $w$ takes value $2,1,0$\\
$w^2$ can take $4,1,0$ at different possibilities\\
$<w^2> = 4*\frac{1}{4} + 1*\frac{1}{2} + 0*\frac{1}{4} = 1.5$\\
So:\\
$\sigma w = \sqrt{<w^2> - <w>^2} = \sqrt{0.5}$\\
\newline

\textbf{5.9}\\
(a)vector\\
(b)vector\\
(c)scalar\\
(d)scalar\\
(e)vector\\
(f)scalar\\
\newline

\textbf{5.10}\\
$A = (3,-2,6), B = (-1,4,4)$\\
$|A| = 7, |B| = \sqrt{33}$\\
$A+B = (-2,3,10), A-B = (4,-6,2)$\\
$AB = 13$\\
$A*B = (-32,-18,10)$\\
$cos\theta = \frac{AB}{|A||B|} = \frac{13}{7\sqrt{33}}$\\
\newline

\textbf{5.11}\\
two diagonal lines are $(1,1,1),(-1,1,-1)$\\
$cos\theta = \frac{-1}{3}$\\
\newline

\textbf{5.13}\\
$grad\, f = \overrightarrow{i}(4x-5yz)+\overrightarrow{j}(-5xz)+\overrightarrow{-5xy+2z}$\\
$\nabla f = 4+0+2 = 6$\\
\newline

\textbf{5.14}\\
(a) $div\cdot grad\, g(x,y,z) = div \dot (\overrightarrow{i}(\frac{\partial}{\partial x})+\overrightarrow{j}(\frac{\partial}{\partial y})+\overrightarrow{k}(\frac{\partial}{\partial z}))$\\
$\frac{\partial^2}{\partial x^2} + \frac{\partial^2}{\partial y^2} + \frac{\partial^2}{\partial z^2}$\\
(b)$div \cdot \overrightarrow{r} = (\overrightarrow{i}\frac{\partial}{\partial x} + \overrightarrow{j}\frac{\partial}{\partial y} + \overrightarrow{k}\frac{\partial}{\partial z})(\overrightarrow{i}x + \overrightarrow{j}y + \overrightarrow{k}z) = 1+1+1 = 3$\\
\newline

\textbf{5.15}\\
(a)$|B| = \sqrt{13}$\\
(b) with each axis:\\
$cos\alpha = \frac{(3,-2,0,1)(1,0,0,0)}{\sqrt{13}} = \frac{3}{\sqrt{13}}$\\
$cos\alpha = \frac{(3,-2,0,1)(0,1,0,0)}{\sqrt{13}} = \frac{-2}{\sqrt{13}}$\\
$cos\alpha = \frac{(3,-2,0,1)(0,0,1,0)}{\sqrt{13}} = 0$\\
$cos\alpha = \frac{(3,-2,0,1)(0,0,0,1)}{\sqrt{13}} = \frac{1}{\sqrt{13}}$\\
\newline

\textbf{5.16}\\
(a) no\\
(b) yes\\
(c) yes\\
(d) yes\\
$[\^{L}_x^2,\^{L}^2] = \^{L}_x[\^{L}_x,\^{L}^2] + [\^{L}_x,\^{L}^2]\^{L}_x = 0$\\
\newline

\textbf{5.18}\\
$[\^{L}_x^2,\^{L}_y] = \^{L}_x[\^{L}_x,\^{L}_y]+ [\^{L}_x,\^{L}_y]\^{L}_x$\\
$=i\hbar(\^{L}_x\^{L}_z+\^{L}_z\^{L}_x)$\\
\newline

\textbf{5.19}\\
coordinate conversion\\
$(x,y,z) -> (r,\theta,\phi)$\\
$r = \sqrt{x^2+y^2+z^2}$\\
$\theta = cos^{-1}(\frac{z}{r})$\\
$\phi = tan^{-1}(\frac{y}{x})$\\
(a) given$(1,2,0)$\\
$r = \sqrt{5}$\\
$\theta = \frac{\pi}{2}$\\
$\phi = tan^{-1}2$\\
(b) given $(-1, 0, 3)$\\
$r = \sqrt{10}$\\
$\theta = cos^{-1}(\frac{3}{\sqrt{10}})$\\
$\phi = tan^{-1}0 = \pi$\\
(c) given $(3,1,-2)$\\
$r = \sqrt{14}$\\
$\theta = cos^{-1}(\frac{-2}{\sqrt{14}})$\\
$\phi = tan^{-1}(\frac{1}{3})$\\
(d) given $(-1,-1,-1)$\\
$r = \sqrt{3}$\\
$\theta = cos^{-1}(\frac{-1}{\sqrt{3}})$\\
$\phi = tan^{-1}(1) = \frac{5}{4}\pi$\\
\newline

\textbf{5.20}\\
Coordinate conversion:\\
$x = rsin\theta cos\phi$\\
$y = rsin\theta sin \phi$\\
$z = rcos\theta$\\
(a) given $r = 1, \theta = \pi/2, \phi = \pi$\\
$x = -1, y = 0, z = 0$\\
(b) given $r = 2, \theta = \pi/4, \phi = 0$\\
$x = \sqrt{2}, y = 0, \sqrt{2}$\\
\newline

\textbf{5.21}\\
(a) sphere\\
(b) cone\\
(c) plane perpendicular to xy plane\\
\newline

\textbf{5.22}\\
the volume is:\\
$V = \int_0^{2\pi}\int_0^{\pi}\int_0^r r^2sin\theta drd\theta d\phi = \frac{4}{3}\pi r^3$\\
\newline

\textbf{5.23}\\
the shape of $\overrightarrow L$ is a cone\\
So the eigenvalue of $c = \overrightarrow L_z = m\hbar$\\
the eigenvalue of $b = \overrightarrow L^2 = l(l+1)\hbar^2$\\
$cos\theta = \frac{c}{\sqrt{b}} = \frac{m}{\sqrt{l(l+1)}}$\\
given $l = 2$\\
$m = -2,-1,0,1,2$\\
then $cos \theta = \frac{-2}{\sqrt{6}}, \frac{-1}{\sqrt{6}}, 0, \frac{1}{\sqrt{6}}, \frac{2}{\sqrt{6}}$\\
\newline

\textbf{5.24}\\
Given $cos\theta = \frac{c}{\sqrt{b}} = \frac{m}{\sqrt{l(l+1)}}$\\
when $m = -l, -l+1, \dots, l-1, l$\\
when $m = l$, $cos^2\theta = \frac{l}{l+1}$\\
when $l -> \inf$ then $cos\theta -> 1, \theta = 0$\\
\newline

\textbf{5.28}\\
$\^{L}^2Y^0_2 = 2*(2+1)Y^0_2$\\
\newline

\textbf{5.29}\\
$\^{L}_z^3Y^m_l = (m\hbar)^3Y^m_l$\\
\newline

\textbf{5.30}\\
$\^{L}_x^2+\^{L}_y^2 = \^{L}^2 - \^{L}_z^2$\\
So the eigenvalue is $l(l+1)\hbar^2 - m^2\hbar^2$\\
\newline

\textbf{5.31}\\
(a) given $l = 2$, $m = -2,-1,0,1,2$\\
the measured value $m\hbar = -2\hbar, -\hbar, 0, \hbar, 2\hbar$\\
(b) given $l = 3$\\
the measured value $m\hbar = -3\hbar,-2\hbar, -\hbar, 0, \hbar, 2\hbar, 3\hbar$\\
\newline

\textbf{5.32}\\
similar to $\^{L}_z$, $[\^{L}^2,\^L_y] = 0$\\
the eigenvalue is also $-\hbar, 0, \hbar$\\
\newline

\textbf{5.33}\\
the other parameters are not important since it is time term\\
Given $Y^1_2$\\
the measurement of $\^{L}^2 = l(l+1)\hbar = 6\hbar$\\
the measurement of $\^{L}_z = m\hbar = \hbar$\\
\newline

\textbf{5.35}\\
$\^{L}_{-}Y_1^1 = Y_1^0$\\
$\^{L}^2_{-}Y_1^1 = Y_1^{-1}$\\
$\^{L}^3_{-}Y_1^1 = 0$\\
\newline

\textbf{5.37}\\
(a) true\\
(b) false\\
(c) true\\
(d) true\\
(e) true\\
(f) false\\
it is possible to find an eigenfunction, but not a set of eigenfunction\\
\newline

\end{document}