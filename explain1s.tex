\documentclass{article}
\usepackage{graphicx} % Required for inserting images
\usepackage{CJK}
\usepackage{amsmath}
\usepackage{mathtools}
\title{Quantum Chemistry}
\author{LuMg}
\date{Aug 2023}

\begin{document}

\maketitle
Explain the shape of electron movement.\\
\begin{equation}
      \nabla ^2\Psi = \frac{1}{c^2}\frac{\partial^2 \Psi}{\partial t^2}
\end{equation}
\begin{equation}
\nabla^2\phi +\frac{2m}{\hbar^2}(E-V)\phi = 0
\end{equation}
\begin{equation}
\hat{H}\Psi = i\hbar\frac{\partial\Psi}{\partial t}
\end{equation}
\begin{equation}
    \frac{\partial^2\phi_{tr}}{\partial X^2}+\frac{\partial^2\phi_{tr}}{\partial Y^2}+\frac{\partial^2\phi_{tr}}{\partial Z^2}+\frac{2M}{\hbar^2}E_{tr}\phi_{tr} = 0
\end{equation}
\begin{equation}
    \phi_T = \phi_{tr}(X,Y,Z)\phi(x,y,z)
\end{equation}
\begin{equation}
    \nabla^2\phi + \frac{2\mu}{\hbar^2}(E+\frac{Ze^2}{r})\phi = 0
\end{equation}
\begin{equation}
    \phi = \phi(r,\theta,\phi) = R(r)\Theta(\theta)\Phi(\phi)
\end{equation}
\begin{equation}
    \frac{d^2\Phi}{d\phi^2}+m^2\Phi = 0, m = 0,\pm1,\pm2\dots
\end{equation}
\begin{equation}
    \Phi_m(\phi) = \frac{1}{\sqrt{2\pi}}exp(im\phi)
\end{equation}
\begin{equation}
    \Theta_{lm}(\theta) = (-1)^{\frac{m+|m|}{2}}\sqrt{\frac{(2l+1)(l-|m|)!}{2(l+|m|)!}}P_l^|m|(cos\theta)
\end{equation}
\begin{equation}
    R_{nl}(r) = -\sqrt{\frac{2Z}{na_0}^3\frac{(n-l-1)!}{2n((n+l)!)^3}}exp(-\rho/2)\rho^lL^{2l+1}_{n+l}(\rho)
\end{equation}
\begin{equation}
    \rho = \frac{2Z}{na_0}r, a_0 = \frac{\hbar^2}{\mu e^2}
\end{equation}
\begin{equation}
\begin{array}{l}
n = 1,2,3,4\dots \\
     l = 0,1,2,3,\dots,n - 1 \\
     m = 0, \pm1,\pm2,\dots,\pm l
\end{array}
\end{equation}
\begin{equation}
    \phi_{1s} = \frac{1}{\sqrt{\phi}}(\frac{Z}{a_0})^{\frac{3}{2}}e^{-\frac{Zr}{a_0}}, n=1,l = 0,m=0
\end{equation}
\end{document}